
\section{Related Work}

Previous work on designing spatial processor focuses on the hardware architecture of the processor, while the optimization of the memory system is only partially taken into account.
In \cite{parashar2014efficient} a spatial processor with distributed control across PE using triggered instructions is presented. The design is not tailored for a specific set of applications and they do not perform analysis on the interactions between the memory and processing systems, leaving the modeling of the memory system as future work.

Plasticine, a spatial processor optimized for the acceleration of parallel patterns is presented in \cite{prabhakar2017plasticine}. Whitin their architecture, the number of memory units (PMUs) and processing units (PCUs), and their interconnections are not optimized around specific applications.

In \cite{budiu2004spatial}, a framework to generate Application Specific Hardware (ASH) from a C application is presented. To handle concurrent memory requests the design uses a hierarchy of busses and arbiters, which creates a bottleneck. This means that their memory system is overwhelmed because it is not tailored for the PEs it uses.

Spatially distributed PEs with a dedicated configuration register allow to configure the PEs to one of the operating modes at compile time in \cite{streamproc2019} . However, there is no automated design flow to efficiently map algorithms to the processor architecture.

An interesting approach is used in Catena\cite{cerqueira2020catena}, where multiple techniques - \textit{clock gating}, \textit{power gating} and \textit{voltage boosting} - are applied in a fine-grained way to optimize energy efficiency. These techniques can be used to explore the power/latency tradeoff of specific applications. However, the impact of the memory system on the performance of the design is not modeled and the memory system is not co-designed with the spatial processor, potentially resulting in an inefficient utilization of the hardware resources; moreover, Catena lacks high-level language support.
