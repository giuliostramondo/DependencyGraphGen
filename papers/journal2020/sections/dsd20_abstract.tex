\begin{abstract}

Spatial processor architectures are essential to meet the increasing demand in performance and energy efficiency of both embedded and high performance computing systems.  Due to the growing performance gap between memories and processors, the memory system often determines the overall performance and power consumption in silicon. The interdependency between memory system and spatial processor architectures suggests that they should be co-designed. For the same reason, state-of-the-art design methodologies for processor archiectures are ineffective for spatial processor architectures because they do not include the memory system. In this paper, we present \frameworkname: an automated framework for co-design-space exploration of spatial processor architecture and the memory system, starting from an application description in a high-level programming language. In addition, we propose a spatial processor architecture template that can be configured at design-time for optimal hardware implementation. To demonstrate the effectiveness of our approach, we show a case-study of co-designing a spatial processor using different memory technologies.

%Application-specific hardware is an effective way to face the increasing demand in performance and power. There is a wide range of choices to consider throughout the design and implementation phases. In this paper, we define a space of application-specific hardware generated from an input application and a methodology to explore these solutions. We demonstrate empirically that these solutions can be implemented in hardware using our hardware templates. Our approach allows the exploration of hardware design parameters and enables the analysis of latency, area and power tradeoff. To demonstrate the potential of our approach we assess the benefit of Magnetoresitive RAM over Static Ram for a use case application.
\end{abstract}
