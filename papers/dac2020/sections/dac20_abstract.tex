\begin{abstract}

Custom processor architectures are essential to meet the increasing demand in performance and power consumption of embedded and high performance computing systems. Due to the massive amount of data handling by the processors, the memory system determines the overall performance and power consumption in silicon. Since the memory system and the processing system are interdependent, they must be co-designed. State-of-the-art tool flow for custom processor design-space exploration focus on processor architecture optimization but does not include memory system. In this paper, we present \frameworkname: an automated framework for co-design-space exploration of custom processor architecture and memory system starting from an application description in a high-level programming language. In addition, we propose an spatial processor architecture template that can be configured at design-time for optimal hardware implementation. To demonstrate the effectiveness of our approach, we show a case-study of co-designing a custom-processor architecture using different memory technologies.

%Application-specific hardware is an effective way to face the increasing demand in performance and power. There is a wide range of choices to consider throughout the design and implementation phases. In this paper, we define a space of application-specific hardware generated from an input application and a methodology to explore these solutions. We demonstate empirically that these solutions can be implemented in hardware using our hardware templates. Our approach allows the exploration of hardware design parameters and enables the analysis of latency, area and power tradeoff. To demonstrate the potential of our approach we assess the benefit of Magnetoresitive RAM over Static Ram for a use case application.
\end{abstract}

