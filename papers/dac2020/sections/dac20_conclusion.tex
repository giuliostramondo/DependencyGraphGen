\section{Conclusion and Future Work}
We presented \frameworkname~a framework for the design space exploration of  memory-aware custom processors. Our framework allows to automatically design an application-specific processor using two levels of memories. The framework enables the design space exploration of different hardware technologies and \textit{co-designs} the memory system and the custom processor. We have empirically demonstrated the implementability of the generated architecture using our functional unit hardware templates. Lastly, we have shown the capabilities of the framework using three case studies, where we where able to obtain quantitative insights over the use of the MRAM technology against SRAM for a use case application. We were able to conclude that for a matrix vector multiplication 10x10 the most energy efficient architecture generated with \frameworkname~with MRAM L2M has \textbf{45\% higher latency than the best SRAM counterpart, with 25\% decrease in power consumption}. We did not obtain the same benefit for a matrix matrix multiplication 10x10, where the MRAM architecture obtained only a \textbf{2.2\% improvement in energy consumption with a 3\% slowdown in comparison to the most energy efficient SRAM architecture}. Moreover, we concluded that for a matrix vector multiplication \textbf{the benefit in energy consumption given by MRAM decrease as the matrix dimension increase, due to the larger amount of write operations which have high energy consumption.}
We are currently working on automating the generation of the RTL implementation of the architectures generated by \frameworkname~using the presented hardware templates. In the near future, we plan to enhance our framework adding the capability of automatically \textit{merging} multiple application-specific architectures, generating \textit{multi-application application-specific hardware}.
